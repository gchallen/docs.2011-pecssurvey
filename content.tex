\begin{flushenumbf}

\item \textbf{How would you define PeCS? What makes PeCS distinct from
related fields, such as cyber-physical systems?}

Discussing ``pervasive'' systems ``at scale'' would seem to be redundant:
anything truly pervasive is by definition deployed at scale. However, the
redundancy embedded in PeCS highlights scale as a core challenge currently
arresting the pervasive computing vision. So I would define ``Pervasive
Computing at Scale'' as a continuation of earlier pervasive computing visions
while acknowledging scale as an exciting current research challenge.

What distinguishes PeCS from cyber-physical systems (CPS) and an earlier
community interest in wireless sensor networks (WSNs) is that these
technologies, while pervasive in nature, tend to be explored outside of a
human context. The scheduled program and discussions that occurred at the
PeCS workshop indicate an interest within the community of exploring the
interaction between humans and pervasive computing and incorporating these
observations into the design of future pervasive systems.

\item \textbf{What aspects of PeCS are already well understood by the
community?}

As a newcomer to this field, I don't feel qualified to answer to fully answer
this question. At the workshop most topics seemed up for discussion. I did
get a strong sense of vision from those that had been working in this area
for longer, so it seems that the community has a strong sense of direction.
However, as human experience with different technologies changes and large
numbers of people vote for different approaches with their money and
attention, it is clear that this human experience should lead to revising the
original pervasive computing blueprint.

\item \textbf{What are the Grand Challenges in PeCS and what are the road
blocks for each of these challenges?}

The composition of this ``Grand Challenge'' list is based on my own interests
and does not pretend to reflect the totality of such challenges in pervasive
computing:

\begin{enumerate}

\item \textbf{Eliminate non-mobile personal computing.} We can already
imagine a world where desktop computers and even laptop-class machines become
obsolete, replaced by handheld devices interacting with powerful cloud
resources and harnessing computation and energy from the environment around
them. Achieving this vision requires improving the capabilities, interfaces,
and battery lifetime of portable devices while improving their interaction
with the cloud.

\item \textbf{Use data analytics to change our lives.} The search
functionality provided by Google and others is ill-suited to allowing users
to cope with the ever-expanding universe of data. Search is ``pull-based'',
whereas new pervasive systems will route relevant data to users in a
``push-based'' fashion. Pervasive computing is required to help provide the
context and information about users to enable this data delivery to succeed,
as well as helping drive data analytics that transform data into
user-initiated action.

\item \textbf{Understand how computers transform energy into information.}
Pervasive computing platforms are likely to be limited by several factors,
including human attention and available energy. Today's computers transform
energy into information but while energy can be easy to measure the relevance
and utility of the information produced is much harder to quantify, making
the fundamental energy efficiency ratio difficult to define or measure.
However, improving the energy usage of computing systems requires measuring
and optimizing this ratio, leading to the requirement that we better
understand the energy transformation process.

\item \textbf{Integrate computation and storage into the environment.} Today,
people tend to go where computers are, in order to use desktops, or computers
tend to go where people are, traveling in pockets and laptop bags. In the
future, computation, storage and other computing resources may be embedded in
the environment in places where they can serve a broad user community. This
will reduce the requirement that future users carry computation on their
persons, while moving computation closer to sources of data and points of
control. It may also allow future pervasive devices to harness environmental
energy sources such as solar power and convert them directly to computation,
rather than being attached to the grid.

\end{enumerate}

\item \textbf{What interdisciplinary collaboration would be critical to meet
the grand challenges? What are the major challenges and proposed solutions
for interdisciplinary research?}

I see three specific areas of future interdisciplinary work:

\begin{enumerate}

\item \textbf{With computer scientists studying human-computer interaction.}
This is perhaps the most obvious area for collaboration, and the easiest
since these are colleagues in our own field. Also, as evidenced by the PeCS
Workshop itself, many HCI specialists are already doing working in pervasive
computing or doing work with broad intersections.

\item \textbf{With social scientists.} Reaching out to social scientists is
important for two reasons. First, large-scale understanding of human behavior
is critical to designing systems that humans will actually use. Second, as we
design and deploy new pervasive systems there are opportunities to improve
our understanding of human behavior by feeding the data they collect to
those in fields that study relevant problems.

\item \textbf{With engineers that design and build existing infrastructure.}
When we talk about embedding computing in the environment, the environment is
a built environment and it is built by engineers and specialists in other
areas. We need to build connections with those that build our buildings,
roads, power grids, and other key infrastructure components in order to
understand how we can integrate pervasive computing into these existing and
complex systems.

\end{enumerate}

\item \textbf{What kinds of technology solutions can be expected to evolve
out of PeCS that will create millions of jobs? How would you recommend to
facilitate technology transfer of research investments in this area?}

If I knew how to answer the first half of this question, I would go out and
make sure those millions of people were working for me! I think it's
near-impossible to foresee the kind of return on investment that research in
computer science can achieve, but would also point out that our field has a
very good track record in creating and advancing technologies like personal
computers and the Internet that have defined entire industries.

I think that active and open exploration of the pervasive computing design
space in academia will help industry avoid local-maxima in design and
implementation. Technology transfer will be facilitated by the academic
publishing culture and accelerate by the contacts between academia and
industry necessary for large-scale pervasive computing research to succeed.

\item \textbf{Why do you think the federal government should support PeCS
research? Why couldn't today's industry achieve these goals? Suggest specific
funding agencies and programs that will effectively support the next
generation of PeCS research.}

I believe that industry will continue to have a large role in actually
implementing and monetizing pervasive computing technologies. But their
ability to do so is largely dependent on the capabilities of this and the
next generation of computer scientists which we as academics are training.
Cutting edge research at Universities can help excite students and lure them
into this area while training them to make sure that they can seamlessly
transfer into industry positions as many of them ultimately will.

NSF CISE has a major role to play in funding this area. There are likely
defense-related activities of these technologies that overlap non-defense
applications and could be funded by the Defense Department. E-ARPA and the
Department of Energy could fund pervasive computing applications that improve
our ability to understand and control societal energy consumption.

\item \textbf{What are the educational opportunities and challenges for PeCS,
including multidisciplinary education and training of faculty and postdocs?}

Our primary goal and opportunity is to use changes in the pervasive landscape
to alter how we teach students. Pervasive computing integrates many aspects
of technology originally developed separately, and offers us the chance to
develop new integrative approaches that both excite and prepare future
engineers.

If the NSF establishes the field with new funding programs, I believe that
new faculty will choose to center their research in this area. At the
workshop it was clear that, at least recently, few faculty members would
consider ``pervasive computing'' to be their core area. (It seems like this
was a more common area to build a research program around 10 years ago.)

\item \textbf{What are the short term (3-5 years), mid term (5-10 years), and
long-term visions and goals of the PeCS field?}

I would submit my ``Grand Challenge'' suggestions as long-term visions and
goals.

Mid-term goals include better user interfaces and more capable mobile devices
that begin to obsolete desktop machines and attention to the interface
between mobile devices and the cloud. We also need to determine how to best
teach these concepts to students and evaluate how well we can prepare them to
program in a pervasive world.

Short-term goals include better energy profiling on mobile devices and the
development of a set of visionary applications that will help drive
innovation forward. As presented in the workshop, there are opportunities in
personal assistance, productivity, health care and other areas where our
efforts will both intersect with pervasive computing goals and benchmarks
while also spurring commercial innovation.

\item \textbf{What were the top three ideas emerging from the workshop that
most interested you?}

\begin{enumerate}

\item \textbf{Interest in testbeds.} Given our whitepaper, I was pleased that
there was a general sense that testbeds are critical to moving the pervasive
vision forward.

\item \textbf{Exploring new mobile interfaces.} The workshop highlighted for
me the role the interfaces and usability are going to play in advancing
pervasive computing. With smaller, more capable devices meeting bigger, more
personal data, deciding how devices present and facilitate interaction with
information is critical.

\item \textbf{Embedding computing into the environment.} I am very interested
in a future where computation migrates from being attached to our bodies and
walls and instead perpetually-powered and integrated into our environment.

\end{enumerate}

\item \textbf{What are ways that NSF and the community can do to advance the
ideas discussed at the workshop? What activities would you propose to build a
stronger PeCS community?}

First, NSF should provide funding for the PeCS community and supporting
integrative PeCS-related activity. A new program could be structured along
the lines of the existing cyber-physical systems program, which divides
research activities into foundations, tools, and systems. After attending the
PeCS workshop I believe that there are a set of distinct research activities
that would be funded by this program and not by other related programs.

Second, the NSF should facilitate interaction between computer scientists and
industry players that can provide access to hardware, software, and data sets
taken from large numbers of real phones and other pervasive devices.

Finally, the NSF should fund educational outreach and activities. Pervasive
computing has a great deal of exciting potential through its potential for
making data for available and useful, and in the way that it engages human
users.

\end{flushenumbf}
